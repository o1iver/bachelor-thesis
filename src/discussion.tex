\chapter{Discussion}
\label{discussion}

The results described in chapter~\ref{results} offer an interesting perspective on the usefulness of this approach for the transformation of Simulink models into Markov Decision Processes. This short discussion aims to provide some opinions as to the reasons for some of the positive and negative results seen in chapter~\ref{results}. The opinions here are based on experience with other extraction models and knowledge of certain implementation difficulties. They are not based on facts and may, in fact, be wrong. Nonetheless they try to serve as a basis for potential improvements of the extraction algorithm.

The approach described in the last few chapters does, indeed, seem to have some merit. Although the MDP's simulated responses have a strong variance and the correlation between distributions is extremely low, the responses do indicate that a certain behavioural transfer has occured. Especially promising are the simulated reponses' means, which definitely indicate similar system reactions for both disturbances.

The fact that the system reacts similarily even though the disturbances occur at different times indicates that the transition probabilities have not been strongly disturbed by the biconditionality assumption discussed in section~\ref{subsec:stateoutput}. This may however prove to be a false conclusion. The effect that the order of extraction simulations has on the resulting system dynamics must be studied in more depth. This is because the order of discovery may have an effect on the accuracy of the extracted model, because the increasing order of the action set may lead to the discovery of states during an `increasing' system reaction, whilst a decreasing order of the action set may lead to more state discoveries during a `decreasing' system reaction. A simple experiment would entail the comparison of systems extracted from an identical source model, but with inverted action sets.

The low correlation of distributions in the state space may not be a valuable metric. Reasons for this include the fact that the correlation decreases with an increasingly large state space, even though the underlying system dynamics may remain equally accurate. Additionaly a low state distribution correlation does not necessarily prove that the Simulink and the MDP system behave differently, only, that the distributions over the state space do not correlate \textit{for each epoch}. The distribution correlation for both input signals, but especially the scaled step input, described in sections~\ref{sec:respconst} and~\ref{sec:respstep}, can, in all likelihood, also be increased by increasing the number of probabilistic simulations for each source-state/action tuple (see sections~\ref{sec:probabilisticsimulation} and~\ref{sec:resultsextraction}). The number of probabilistic extractions was set to only 20. A much greater number of simulations is required in order to correctly extract the nature of the random input's distribution. An analysis of the system responses after an extraction with 500 or 1000 probabilistic simulations, is likely to show better results.

The implementation of this extraction algorithm and the study of its strenghts and weaknesses have shown that, although the approach shows promise, the effect different extraction parameters have on the quality of the extracted model must be studied in more depth before these are used for serious optimization. Nonetheless, it must be remembered, that the only important metric of success is the usefulness of an extracted model for optimization. The advantages of MPD-/POMDP-based optimization may easily outweigh the loss of accuracy an extraction entails. Further study of this approach must take into account the effect of extraction errors on the quality of the final optimal policy.